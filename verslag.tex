\documentclass[a4paper,11pt]{report}
\usepackage[T1]{fontenc}
\usepackage[utf8]{inputenc}
\usepackage{lmodern}

\title{Compiler Construction}
\author{Sander in 't Veld \and Joran van Apeldoorn}

\begin{document}

\maketitle
\tableofcontents

%\begin{abstract}
%\end{abstract}

\chapter{Compiler Phases}
The first three phases, `Loading', `Context Analysis' and `Type Checking', interpret the code and build the full abstract syntax tree. The next three phases restructure, optimise and prepare the AST, before the last phase, `Code Generation'. ZINNEN ZIJN LASTIG.

\section{Loading the code}
\subsection{Running the preprocessor}
Our compiler runs the \texttt{CPP} preprocessor and saves the resulting code in a hidden file called `\texttt{dir/.code.cvc.cpp}', if the original code was called `\texttt{dir/code.cvc}'. This step removes single- and multi-line comments, while retaining line and column numbers. It also links header files. The flag \texttt{-I path} can be used to search for header files in the directory \texttt{path}.

The resulting code is used as input for the next phase.

\subsection{Scanning}
Our scanner interprets the \texttt{\#line} preprocessor hints to provide accurate error messages during scanning and parsing.

The scanner accepts \texttt{CR}, \texttt{LF}, \texttt{CR LF} and \texttt{LF CR} as a valid line ending.

Integers between $-2^{31}$ and $2^{31}-1$ (inclusive) are always supported; larger integers may be supported, depending on the platform. Floats have similar ranges of support. The scanner accepts both \texttt{.5} and \texttt{2.} as shorthand notation for \texttt{0.5} and \texttt{2.0} respectively.

\subsection{Parsing}

\section{Context Analysis}
Namelinking happens before declaration splitting.
\subsection{Namelinking}
\subsection{Declaration splitting}

\section{Type Checking}
\subsection{Type Inference}
\subsection{Type Matching}

\section{Code Restructuring}
\subsection{Iterator Splitting}
\subsection{Array Splitting}
\subsection{Getters and Setters}
\subsection{Ternary Operators}

\section{Code Optimisation}
\subsection{Constant Calculating}

\section{Code Preparation}
\subsection{Variable Counting}
\subsection{Nested Function Renaming}

\section{Code Generation}

\end{document}
